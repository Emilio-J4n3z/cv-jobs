%%%%%%%%%%%%%%%%%%%%%%%%%%%%%%%%%%%%%%%%%
% Modern Curriculum - José Emilio Sánchez Juárez
% Cybersecurity Researcher & NLP Specialist
%%%%%%%%%%%%%%%%%%%%%%%%%%%%%%%%%%%%%%%%%

\documentclass[letterpaper]{moderncurriculum}
\usepackage{hyperref}

\begin{document}
\header{José Emilio Sánchez Juárez}{Investigador en Ciberseguridad \& NLP | En Formación Continua}

%%%%%%%%%%%%%%%%%%%%%%%%%%%%%%%%%%%%%%%%%%
% SIDEBAR CONFIGURATION
%%%%%%%%%%%%%%%%%%%%%%%%%%%%%%%%%%%%%%%%%%

\birth{Cuautitlán Izcalli, Estado de México}
\address{https://github.com/Emilio-J4n3z}
\phonenumber{+52 561-861-4669}
\site{https://emilio-j4n3z.github.io}
\linkedin{https://linkedin.com/in/josé-emilio-sánchez}
\mail{emilinho18@gmail.com}

\profile{}

%----------------------------------------------------------------------------------------
% TECHNICAL KNOWLEDGE - Optimizado y expandido
%----------------------------------------------------------------------------------------

\knowledge{%
{Python,} {C/C++,} {Java,} {Bash,} {SQL,} {JavaScript,} {HTML/CSS,} {LaTeX,} {FastAPI,} {Flask,} {Docker,} {Git/GitHub,} {Linux}
}

%----------------------------------------------------------------------------------------
% TECHNICAL TOOLS - Categorizado y completo
%----------------------------------------------------------------------------------------

\technicaltools{
\textbf{Security:} Nmap • Burp Suite • Metasploit • Wireshark • SQLMap • Hashcat • John the Ripper • Ghidra \newline
\textbf{Development:} Git/GitHub • Docker • VS Code • Jupyter • MySQL Workbench \newline
\textbf{NLP/ML:} NLTK • spaCy • transformers • scikit-learn • TensorFlow
}

%----------------------------------------------------------------------------------------
% SOFT SKILLS
%----------------------------------------------------------------------------------------

\softsklls{
- Pensamiento Analítico \& Estratégico \newline
- Resolución Creativa de Problemas \newline
- Comunicación Técnica Clara \newline
- Mentoría \& Liderazgo Comunitario \newline
- Resiliencia \& Aprendizaje Continuo \newline
- Trabajo Colaborativo Open Source
}

%----------------------------------------------------------------------------------------
% LANGUAGES
%----------------------------------------------------------------------------------------

\languages{
\textbf{Español:} Nativo \newline
\textbf{Inglés:} Intermedio (B1) \newline
\textbf{Portugués:} Básico (A2)
}

\sidebar

%%%%%%%%%%%%%%%%%%%%%%%%%%%%%%%%%%%%%%%%%%
% MAIN BODY
%%%%%%%%%%%%%%%%%%%%%%%%%%%%%%%%%%%%%%%%%%

%----------------------------------------------------------------------------------------
% PROFILE PROFESIONAL
%----------------------------------------------------------------------------------------

\section{Perfil Profesional}
\vspace{0.1pt}
Investigador en ciberseguridad en formación continua con profundo interés en criptografía, análisis de malware y explotación de binarios (buffer overflow, pwning). Mi aproximación integra seguridad ofensiva mediante competencias CTF (web exploitation, SSTI, criptografía) con mentalidad defensiva de blue team, anticipando movimientos del adversario con pensamiento estratégico similar al ajedrez.

Especializado en procesamiento de lenguaje natural (NLP) aplicado a la preservación de lenguas indígenas y desarrollo de sistemas de traducción automática. Actualmente desempeñando investigación en IIMAS UNAM desarrollando módulos de análisis automatizado de textos legales con técnicas avanzadas de NLP. Mi trabajo se inspira en pioneros como Ken Thompson ("Reflections on Trusting Trust"), Kevin Mitnick y Bruce Schneier, buscando democratizar el conocimiento de ciberseguridad mediante contenido técnico accesible en español. Comprometido con el aprendizaje constante y la contribución a comunidades open source en seguridad y tecnologías lingüísticas.
\vspace{0.1cm}

%----------------------------------------------------------------------------------------
% PROFESSIONAL EXPERIENCE
%----------------------------------------------------------------------------------------

\section{Experiencia Profesional}
\vspace{-8pt}

\begin{biglist}
    \biglistitem{2025-Presente}
    {Asistente de Investigación - NLP \& Análisis de Documentos Legales}
    {IIMAS UNAM, Ciudad de México}
    {Desarrollo de módulos de análisis de texto para procesamiento automatizado de documentos legales. Construcción de herramientas de visualización con FastAPI y Alpine.js. Implementación de pipelines NLP con spaCy y transformers para extracción de entidades y clasificación de sentencias judiciales.}

    \biglistitem{Anterior}
    {Asistente de Investigación - Ciberseguridad \& Análisis de Vulnerabilidades}
    {CEDETEC LASC, Estado de México}
    {Investigación en ciberseguridad ofensiva y análisis de vulnerabilidades. Aplicación de metodologías de penetration testing y evaluación de protocolos de seguridad. Participación en proyectos de threat assessment y security hardening.}

    \biglistitem{2025-Presente}
    {Investigador en Ciberseguridad \& Creador de Contenido Técnico}
    {Independiente / Comunidad Open Source}
    {Competidor activo en plataformas CTF (PicoCTF, HackTheBox, TryHackMe, EchoCTF) con 15+ writeups publicados. Especialización en web exploitation (SSTI, SQLi), criptografía clásica/moderna y binary exploitation. Miembro del equipo white\_void (CTFtime). Creación de contenido técnico democratizando ciberseguridad en español.}

    \biglistitem{2025-Presente}
    {Mentor en Seguridad \& Líder Comunitario}
    {OffSec Community, Ciudad de México}
    {Mentoría en fundamentos de ciberseguridad, metodologías CTF y análisis criptográfico. Enseñanza de web exploitation avanzada y desarrollo de mindset defensivo (blue team). Facilitación de sesiones colaborativas de investigación en seguridad.}
    
\end{biglist}
\vspace{0.1cm}

%----------------------------------------------------------------------------------------
% EDUCATION
%----------------------------------------------------------------------------------------

\section{Educación}
\vspace{-6pt}
\begin{shortlist}
    \shortlistitem{2022-2026}
    {Ingeniería en Sistemas Computacionales}
    {Tecnológico de Estudios Superiores de Cuautitlán Izcalli}
    {GPA: 91.7/100 • Especialización: Ciberseguridad, Machine Learning, NLP}

    \shortlistitem{2020-2027}
    {Matemáticas Aplicadas y Computación}
    {FES Acatlán - UNAM}
    {GPA: 85.6/100 - Doble titulación simultánea (cursando ambas carreras en paralelo)\newline
    Teaching Assistant: Organización de Computadoras (2021)}
\end{shortlist}
\vspace{0.1cm}

%----------------------------------------------------------------------------------------
% ACHIEVEMENTS & RECOGNITION 
%----------------------------------------------------------------------------------------

\section{Logros \& Reconocimientos}
\vspace{-8pt}
\begin{shortlist}
    \shortlistitem{2025}
    {MeIA 2025 AI Macrotraining Certificate}
    {UNAM}
    {Programa intensivo de Inteligencia Artificial}

    \shortlistitem{2021}
    {2do Lugar - Concurso de Programación Competitiva}
    {M@C Academic Week}
    {Desempeño destacado en resolución algorítmica bajo presión}

    \shortlistitem{2025}
    {Miembro Equipo white\_void CTF}
    {CTFtime Registered}
    {15+ soluciones documentadas • Competencias internacionales de seguridad}

    \shortlistitem{2023-2024}
    {Representante Atlético}
    {FES Acatlán UNAM- TESCI}
    {Representación institucional en competiciones futbolísticas CONDE}

    \shortlistitem{2020-2021}
    {Organizador de Eventos}
    {FLISoL \& Semana Académica}
    {FLISoL (150+ asistentes) • Semana Académica (500+ asistentes)}
\end{shortlist}
\vspace{0.1cm}

%----------------------------------------------------------------------------------------
% PROFESSIONAL DEVELOPMENT
%----------------------------------------------------------------------------------------

\section{Desarrollo Profesional}
\vspace{-8pt}
\begin{shortlist}
    \shortlistitem{2025}
    {MeIA 2025 - Programa de Macroentrenamiento en IA}
    {UNAM}
    {Programa intensivo de Inteligencia Artificial}

    \shortlistitem{2024}
    {Taller de Ética en IA}
    {UNAM}
    {Platicas El rol de la IA en la sociedad - \href{https://eticaia\_unam.gitlab.io/\#}{Ver materiales}}
    
    \shortlistitem{2023}
    {Seminario de Deep Learning}
    {Basado en Curso NYU}
    {Conceptos DEEP LEARNING - Yann LeCun - Alfredo Canziani - \href{https://l52mas.gitlab.io/seminariodl/}{versión ES} / \href{https://atcold.github.io/NYU-DLSP20/}{original EN}}
\end{shortlist}
\vspace{-0.1cm}

%----------------------------------------------------------------------------------------
% TEACHING & SPEAKING
%----------------------------------------------------------------------------------------

\section{Docencia \& Conferencias}
\vspace{-8pt}
\begin{shortlist}
    \shortlistitem{2024}
    {Instructor de Taller NLP - NLP para Principiantes Parte 2}
    {Workshop en Línea}
    {50+ participantes - Python, NLTK, spaCy, transformers - \href{https://www.youtube.com/watch?v=DBzjvICII8k&t=1012s}{Ver video}}

    \shortlistitem{2024}
    {Personal Técnico - Escuela de Verano de Visualización de Datos}
    {Evento Científico}
    {100+ participantes - Soporte técnico visualización científica - \href{https://youtu.be/_xIEXWe93ck}{Ver video}}

    \shortlistitem{2021}
    {Asistente de Enseñanza - Organización de Computadoras}
    {FES Acatlán UNAM}
    {Curso 1111 - Apoyo en arquitectura de computadoras}

    \shortlistitem{2018}
    {Conferencista - Cifrado con GPG}
    {Conferencia FLISoL}
    {Divulgación de criptografía - \href{https://www.youtube.com/watch?v=oZ8eHkTxE4c\&t=25s}{Ver video}}
\end{shortlist}
\vspace{-0.1cm}

%----------------------------------------------------------------------------------------
% RESEARCH INTERESTS - Expandido con contexto
%----------------------------------------------------------------------------------------

\section{Áreas de Investigación}
\vspace{-8pt}

\subsection{Criptografía \& Binary Exploitation}
Análisis de sistemas criptográficos clásicos/modernos, buffer overflows, ROP chains. Inspirado por "Reflections on Trusting Trust" de Ken Thompson.

\subsection{ML Security \& Malware Analysis}
Intersección de machine learning y ciberseguridad: detección de anomalías, análisis automatizado de malware, OSINT potenciado por IA.

\subsection{NLP para Preservación Lingüística}
Preservación digital de lenguas indígenas mediante NLP. Desarrollo de traductores para lenguas de pocos recursos.

\subsection{Blue Team Strategy \& Threat Intelligence}
Mentalidad defensiva anticipando movimientos del atacante. Threat hunting y security hardening.

\end{document}